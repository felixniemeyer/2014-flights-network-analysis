%%
% TUM Corporate Design LaTeX Templates
% Based on the templates from https://www.tum.de/cd
%
% Feel free to join development on
% https://gitlab.lrz.de/tum-templates/templates
% and/or to create an issue in case of trouble.
%
% tum-article class for scientific articles, reports, exercise sheets, ...
%
%%

\documentclass[twocolumn]{tum-article}
%\documentclass[twocolumn, german]{tum-article}
%\documentclass[times, twocolumn]{tum-article}
%\documentclass[times]{tum-article}
%\documentclass{tum-article}

\usepackage{lipsum}

\title{Analysis of the 2014 Passenger Flight Network}
\author{Felix Paul Niemeyer\authormark{1},
	Saicharan Kumar\authormark{2}}
% Author 3\authormark{1}\orcid{0000-0000-0000-0000}}

% if too long for running head
\titlerunning{TUM Article}
\authorrunning{Author 1 et al.}

%\email{niemeyer.felix@tum.de}

\affil[1]{felix.niemeyer@tum.de, MiM}
\affil[2]{saicharan.kumar@tum.de, MiM}

\date{Received: 19.02.2020 - This is work in progress}

\begin{document}

\maketitle

\begin{abstract}
	In this case study we are looking at a dataset about air travel in 2014. We implement some mechanisms to clean the data, as well as mechanisms to enrich this dataset with information we query from wikidata. We estimate some missing data. Starting from there, we analyze some network characteristics for the whole network as well as airline-specific sub-networks. We compare airline-specific networks and find relations between their network's characteristics and their business model. We propose one metric that plays a role in forcasting whether a cooperation between two airlines is advantageous and use this metric to generate an alliances landscape. We compare it's predictive correctness to the real world by looking at the formation of Vanilla Airlines which occured in the year after the one our data stems from. 
\end{abstract}

\section{Introduction}

\lipsum[2]\cite{jirauschek2014}

\section{Theory}

\lipsum[3]

\section{Experimental Setup}

\lipsum[4-5]

\section{Results}

\lipsum[6]

\section{Conclusions}

\lipsum[7]

\section*{Acknowledgements}

\lipsum[8]

\bibliographystyle{IEEEtran}
\bibliography{literature}

\end{document}
